\documentclass[11pt]{article}

\usepackage{amsmath,amssymb}
\usepackage{float}
\usepackage{xcolor}
\usepackage{hyperref}
\usepackage[margin=1in]{geometry}

% Define the invert color command
\newcommand{\invertcolor}{%
    \pagecolor{black}% Set background to black
    \color{white}% Set text color to white
}
% \invertcolor

\title{Reinforcement Learning HW 2}
\author{
Mert Bilgin (7034879) \\
\url{mert.bilgin@student.uni-tuebingen.de}
\and
Lalitha Sivakumar (6300674) \\
\url{lalitha.sivakumar@student.uni-tuebingen.de}
\and
Kevin Van Le (7314700) \\
\url{kevin-van.le@student.uni-tuebingen.de}
}
\date{\today}

\begin{document}
\maketitle

\section{State-Action Value Function and Policy Iteration}

\subsection*{a)}
The state-action value function is given by:

\[
q_{\pi}(s, a) = \sum_{s'} p(s' \mid s, a) \big[ r(s, a, s') + \gamma v_{\pi}(s') \big]
\]


Since we have an undiscounted task, $\gamma = 1$ for the following steps. \\


\noindent For \( q_{\pi}(11, \text{down}) \):

\[
q_{\pi}(11, \text{down}) 
= \sum_{s'} p(s' \mid 11, \text{down}) \big[ -1 + v_{\pi}(s') \big]
\]
Since only \( p(\text{Terminal} \mid 11, \text{down}) = 1 \), we have:
\[
q_{\pi}(11, \text{down}) = -1 + v_{\pi}(\text{Terminal}) = -1 + 0 = -1
\]


\noindent Similarly, for \( q_{\pi}(7, \text{down}) \):

\[
q_{\pi}(7, \text{down}) 
= \sum_{s'} p(s' \mid 7, \text{down}) \big[ -1 + v_{\pi}(s') \big]
\]
Since only \( p(11 \mid 7, \text{down}) = 1 \), we have:
\[
q_{\pi}(7, \text{down}) = -1 + (-14) = -15
\]


\noindent Finally, for \( q_{\pi}(9, \text{left}) \):

\[
q_{\pi}(9, \text{left}) 
= \sum_{s'} p(s' \mid 9, \text{left}) \big[ -1 + v_{\pi}(s') \big]
\]
Since only \( p(8 \mid 9, \text{left}) = 1 \), we have:
\[
q_{\pi}(9, \text{left}) = -1 + (-20) = -21
\]

\subsection*{b)}

We know that the state-value function under a policy $\pi$ is given by:
\[
v_{\pi}(s) = \sum_{a \in \mathcal{A}} \pi(a \mid s) \, q_{\pi}(s, a)
\]

The optimal value can be written in a similar fashion. That is,
\[
v_{*}(s) = \max_{\pi} v_{\pi}(s)
          = \max_{\pi} \sum_{a \in \mathcal{A}} \pi(a \mid s) \, q_{\pi}(s, a)
\]

For the optimal policy $\pi_{*}$, we have:
\[
v_{*}(s) = \sum_{a \in \mathcal{A}} \pi_{*}(a \mid s) \, q_{*}(s, a)
\]

Here, we implicitly switch to $q_{*}$ with $\pi_{*}$, as the optimal action-value can be found greedily by following the path with the most expected reward path. \\

Since the optimal policy selects the action that maximizes $q_{*}(s,a)$,
\[
\pi_{*}(a \mid s) =
\begin{cases}
1, & \text{if } a = \arg\max\limits_{a \in \mathcal{A}} q_{*}(s,a), \\[4pt]
0, & \text{otherwise.}
\end{cases}
\]

Substituting this back, we obtain:
\[
v_{*}(s)
= \sum_{a \in \mathcal{A}}
\Big[ \mathbb{I}\!\left( a = \arg\max\limits_{a' \in \mathcal{A}} q_{*}(s,a') \right) 
\, q_{*}(s,a)\Big]
= \max_{a} q_{*}(s,a) 
\]
where we use $\mathbb{I}$ for the indicator operator.

\subsection*{c)}
We know that for any policy $\pi$, the following holds for the action-value function:
\[
q_{\pi}(s,a) = R_s^a + \gamma \sum_{s' \in \mathcal{S}} P_{ss'}^a \, v_{\pi}(s')
\]

Taking the maximum over all policies, we obtain:
\[
\max_{\pi} q_{\pi}(s,a) = q_{*}(s,a) = 
\max_{\pi} \left( R_s^a + \gamma \sum_{s' \in \mathcal{S}} P_{ss'}^a \, v_{\pi}(s') \right)
\]

Since the rewards $R_s^a$ and transition probabilities $P_{ss'}^a$ do not depend on $\pi$, we can move the maximization inside the sum because the optimum value function achieves the maximum return for every state $s \in \mathcal{S}$.
\[
q_{*}(s,a) = R_s^a + \gamma \sum_{s' \in \mathcal{S}} P_{ss'}^a \, \left( \max_{\pi} v_{\pi}(s') \right)
\]

By definition of the optimal value function $v_{*}(s') = \max_{\pi} v_{\pi}(s')$, we have:
\[
\boxed{
q_{*}(s,a) = R_s^a + \gamma \sum_{s' \in \mathcal{S}} P_{ss'}^a \, v_{*}(s')
}
\]

\subsection*{d)}
An optimal policy can be found by maximizing over the optimal action-value function $q_{*}(s,a)$:

\[
\pi_{*}(a \mid s) =
\begin{cases}
1, & \text{if } a = \arg\max\limits_{a' \in \mathcal{A}} q_{*}(s,a'), \\[6pt]
0, & \text{otherwise.}
\end{cases}
\]

\noindent
It greedily selects the action with the highest estimated return according to $q_{*}$.

\subsection*{e)}

We know that the state-value function is defined as:
\[
v_{\pi}(s) = \sum_{a \in \mathcal{A}} \pi(a \mid s) \, q_{\pi}(s,a)
\]

For any policy $\pi$, the action-value function satisfies:
\[
q_{\pi}(s,a)
= R_s^a + \gamma \sum_{s' \in \mathcal{S}} P_{ss'}^a \, v_{\pi}(s')
\]

Substituting the definition of $v_{\pi}(s')$ into the above equation gives:
\[
q_{\pi}(s,a)
= R_s^a + \gamma \sum_{s' \in \mathcal{S}} P_{ss'}^a 
   \left( \sum_{a' \in \mathcal{A}} \pi(a' \mid s') \, q_{\pi}(s',a') \right)
\]

Simplifying, we obtain the Bellman expectation equation for the action-value function:
\[
\boxed{
q_{\pi}(s,a)
= R_s^a + \gamma \sum_{s' \in \mathcal{S}} \sum_{a' \in \mathcal{A}} 
      P_{ss'}^a \, \pi(a' \mid s') \, q_{\pi}(s',a')
}
\]

\section{Value Iteration}
\subsection{Gridworld: Getting started}

Updated code in agent.py

\subsection{Implement and test value iteration}
\subsubsection*{a)} In \texttt{MazeGrid}, it takes 12 rounds of value iteration until it becomes non-zero because the value of each state is first initialized with 0 and we only receive a reward at the pre-terminal state while taking the optimal action. During value iteration, this state with non-zero reward can only change the neighboring state due to the bellman equation which only considers the current and the next state, therefore for start state to change to a non-zero value it takes the number of states from the start to the pre-terminal states, which is 12.

\subsubsection*{b)} While running the policy for \texttt{BridgeGrid} with the default discount and noise parameters, the agent always avoided crossing the bridge because there was too much risk of falling off and receiving a reward of $-100$. Increasing the discount factor did not make the agent cross either. However, when the noise was set to an extremely small value like  $0.02$, the agent started crossing the bridge to reach the high reward. 

\subsubsection*{c)}

\begin{table}[H]
\centering
\begin{tabular}{|c|c|c|}
\hline
\textbf{Case} & \textbf{Discount} & \textbf{Noise} \\ \hline
a & 0.1 & 0.02 \\ \hline
b & 0.1 & 0.3 \\ \hline
c & 0.4 & 0.02 \\ \hline
d & 0.4 & 0.1 \\ \hline
e & 1 & 0.0 \\ \hline

\end{tabular}
\caption{\texttt{DiscountGrid} parameters for different cases.}
\end{table}

\

\subsubsection*{d)} On \texttt{MazeGrid} with the default parameters and 100 value iteration, we get a start--state value of 0.28. In Exercise 1, we got values of 0.00034 and 0.00221 when running 10 and 10000 episodes respectively. We see a difference between the two values because the first exercise used a random policy where all actions were equally likely. Here, we use value iteration to find the optimal policy and then calculate the value of the start state. The start state value is higher for the optimal policy because the agent is more likely to reach the goal quickly and collect the higher reward.

\subsection{Custom Gridworld Analysis}
The custom gridworld is defined as follows:

\begin{verbatim}
grid = [[' ',' ',' ',+10],
        [' ','#',-10,' '],
        [' ','#',-10,' '],
        [' ','#',+5,' '],
        ['S',' ',' ',' ']]
\end{verbatim}

\noindent \textbf{For discount = 0.9, noise = 0.2, livingReward = 0.0:} From the start state the agent prefers taking the shortest path to the small reward of +5, but if the agent is in the top next state from the start state it prefers the longer path. The reason is the discount factor where the longer path is one step close and the shorter path one step farther away. In the bottom left corner state the agent actually prefers taking the riskier path to the big reward due to the big reward being discounted
less.

\noindent 
\textbf{For discount = 0.9, noise = 0.1, livingReward = 0.0:} For this configuration where the noise is smaller, the agent takes the shortest path to the big reward. The reason is now that due to the smaller noise, the risk is now less to fall into the -10 pits so the best action is now taking that riskier path.

\noindent
\textbf{For discount = 0.9, noise = 0.2, livingReward = 0.5:} When adding a living reward to the default configuration, the agent prefers taking the long safe path to the big reward. Interestingly, on the bottom next state to the small reward, the agent would prefer going the long path to the left to go to the big reward instead of taking the small reward by going a step up. This highlights the effect that the living reward has on the agent, where taking steps is also rewarded.

\noindent
\textbf{For discount = 1.0, noise = 0.2, livingReward = 0.5:} Using the previous configuration with discount = 1.0 is rewarded for being in the run as long as possible. Hence the positive terminal states becomes that states that the agent wants to avoid as it would end the run. If we assume that the top left corner is state (0,0) then the most valuable state is the state (0,2) as this state is the farthest away from all terminal states and the optimal action would be to walk into the wall. The distance to the terminal states still matter here because we have a noise factor where the agent could accidently go to a terminal state.

\noindent
\textbf{For discount = 0.9, noise = 0.4, livingReward = 0.0:} With high noise, the agent wants to avoid the pitfall at all costs, it takes the same route as in the first configuration with a small difference when the state is the bottom right corner. Here, the agent goes to the left to avoid accidently falling into the pit.
\end{document}
