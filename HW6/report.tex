\documentclass[11pt]{article}

\usepackage{amsmath,amssymb}
\usepackage{float}
\usepackage{xcolor}
\usepackage{hyperref}
\usepackage[margin=1in]{geometry}

% Define the invert color command
\newcommand{\invertcolor}{%
    \pagecolor{black}% Set background to black
    \color{white}% Set text color to white
}
% \invertcolor

\title{Reinforcement Learning HW 6}
\author{
Mert Bilgin (7034879) \\
\url{mert.bilgin@student.uni-tuebingen.de}
\and
Lalitha Sivakumar (6300674) \\
\url{lalitha.sivakumar@student.uni-tuebingen.de}
\and
Kevin Van Le (7314700) \\
\url{kevin-van.le@student.uni-tuebingen.de}
}
\date{\today}

\begin{document}
\maketitle

\section{Function approximation}

\subsection*{(a)} Tabular methods are a special case of linear function approximation. For a finite state space with $n$ states, each state $s$ is represented by a one-hot
feature vector $\phi(s) \in \mathbb{R}^n$ defined by
\[
    \phi(s)_i = 
    \begin{cases}
        1, & \text{if } i \text{ corresponds to state } s, \\
        0, & \text{otherwise}.
    \end{cases}
\]

With linear value function approximation,
\[
    \hat{V}(s) = w^\top \phi(s),
\]
the one-hot structure ensures that $\hat{V}(s)$ simply selects the parameter $w_s$. Thus tabular methods are exactly linear function approximators with one-hot features.

\bigskip

\subsection*{(b)} For a continuous state $s = (s_1,\dots,s_k) \in \mathbb{R}^k$, an order-$n$ polynomial feature has the form
\[
    x_i(s) = \prod_{j=1}^k s_j^{c_{i,j}},
\]
where each exponent $c_{i,j}$ is an integer in $\{0,1,\dots,n\}$.  

\noindent Each dimension $j$ therefore has $(n+1)$ possible exponent choices, and the choices across the $k$ dimensions are independent.  
Hence the total number of distinct polynomial features is
\[
    (n+1)^k.
\]

\section{Feature Designing}
A simple feature vector for a linear value function in Super Mario could be

\[
x(s) = (x_1, x_2, x_3, x_4, x_5),
\]

where each entry just picks out something important from the current frame:

\begin{itemize}
    \item $x_1$: Mario's current $x$-position in the level.
    \item $x_2$: his height (ground, platform, or in the air).
    \item $x_3$: distance to the closest enemy in front of him.
    \item $x_4$: whether the next tile is safe to step on (1) or not (0).
    \item $x_5$: his power-up state (small / super / fire).
\end{itemize}

\end{document}
